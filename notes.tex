\documentclass[12pt,a4paper]{article}
\usepackage{fontspec}
\usepackage{polyglossia}
\setmainlanguage{english}
\setotherlanguage{french}
\usepackage{csquotes}
\usepackage{amsmath,amssymb,amsthm}
\usepackage{mathtools}
\usepackage[margin=2cm]{geometry}
\usepackage{fullpage}
\usepackage[backend=biber, style=numeric, maxbibnames=99]{biblatex}
%\bibliography{doc.bib}
\usepackage{hyperxmp}
\usepackage[colorlinks, citecolor=blue]{hyperref}
\usepackage{xcolor}
\usepackage{algorithm2e}
% \usepackage{minted}
% \usepackage[type={CC}, modifier={by-sa}, version={4.0}]{doclicense}
\usepackage{lipsum}

\title{MPRI Graph Algorithms Lecture Notes}
\author{Antonin Décimo}


\def\R{\mathbb{R}}
\def\Z{\mathbb{R}}
\def\N{\mathbb{N}}
\newtheorem{theorem}{Theorem}

\begin{document}
\maketitle

The four-color theorem. (First proposed by Francis Guthrie in October 1852).
Region of every simple map can be colored properly using at most four colors.

\begin{enumerate}
\item Every planar graph (with no loop) can be properly vertex colored using at
most four colors.
\item (Reformulation by Peter Guthrie Tait (1880)). Every bridge-less cubic
  planar graph is 3-edge-colorable.
\end{enumerate}

Exercise. Prove that \(K_{3,3}\) and \(K_5\) are not planar graphs.  Project:
study the best known algorithms to decide if a given graph is planar.

\begin{theorem}
  Statements (1) \& (2) are equivalent.
\end{theorem}

The key idea: 4 colors are elements of \(\Z_2 \times \Z_2\), they are
\((0, 0), (0, 1), (1, 0), (1, 1)\).  3 colors are the non-zero (corresponds to
\enquote{proper}) elements \((0, 1), (1, 0), (1, 1)\).

\begin{proof}
  We may assume \(G\) is a planar graph where every face is a triangle
  (triangulation).
\end{proof}

Some of the main theories developed by the four-color theorem, Tait's
reformulation. Proof of the equivalence and false statements:

Planarity:
\begin{itemize}
\item theory of minor
\item topological minor
\item embedding graphs on surfaces
\item crossing number, genus, …
\end{itemize}

Coloring:
\begin{itemize}
\item chromatic number
\item homomorphism
\item circular, fractional, acyclic, … chromatic number
\item List coloring
\item independence and clique number
\end{itemize}

Edge-coloring:
\begin{itemize}
\item edge-chromatic number
\item fractional edge-chromatic number
\item Hamiltonicity
\item Matching theory
\end{itemize}

A cubic graph has an even number of vertices. The proof comes from the theory of
(nowhere zero) flows.

Exercise. Build a cubic bridgeless graph which is not 3-edge-colorable. Conclude
that this graph is not hamiltonian.

Project. Build an infinite family of planar cubic bridgeless graphs that are not
hamiltonians.

Project. Prove that it is NP-complete to decide if a given cubic bridgeless
graph is 3-edge-colorable.

Project. Prove that every 2-connected graph admits a nowhere zero 6-flow.

Proper coloring.  The chromatic number \(\chi(G)\) is the minimum number of
colors used in a proper vertex coloring of G. Normal colors: \(1, 2, 3, 4, 5, 6,
\dots\)

Greedy algorithm for coloring.  Take an uncolored vertex, assign to it the
smallest color not used by any of its already colored neighbours.

Corollary. \(\chi(G) \leq \Delta + 1\)

Exercice. Equality holds for add-cycles and complete graphs.

Brooks theorem. If \(G\) is a graph not ismorphic to an odd-cycle or to a
complex graph, then \(\chi(G) \leq \Delta\).

Proof (by Lovasz).

\begin{enumerate}
\item We may assume \(\Delta \geq 3\). Towards a proof by induction:
\item \(G\) has no vertex cut or, more generally, no clique-cut.
\item \(G\) is 3-vertex-connected.
\item Considerer a vertex \(v\) of degree \(\Delta\) (\(\Delta \geq 3\)) there
must exists two neighbours \(u, w\) of \(v\) which are not adjacent (why?).
\item \(G \setminus \{u, v\}\) is connected.
\item Order vertices such that \(v_i=u, v_2=w, \dots, v_n=v\) and that for \(i=3
\dots n-1\), vertex \(v_i\) has a neighbour in \(\{v_{i+1}, \dots, v_n\}\).
\item Apply greedy algorith on \(v_1, \dots, v_n\).
\end{enumerate}

Remark on algorithmic application. It can be checked (easily) if \(G\) is
3-connected, then we can use the proof to find a \(\Delta\)-coloring.

Remark. When we are lucky (which means whe have one of the good orderings),
greedy algorithm may produce a coloring using \(\chi(G)\)-colors.

However normally the number of colors used by a greedy coloring is much larger
than \(\chi(G)\).

Bipartite and at least one edge iff \(\chi(G)=2\).

Homework. Find ordering of vertices for interval graphs, for chordal graphs, …

Algorithmic discussion.  Simplest algorithm to decide if \(\chi(G) \leq k\) is
to consider all possible \(k-colorings\), that is \(k^n\) possible
coloring. Better bounds are provided. \(O(2^n n)\) for any
\(k\). \(O(1,3289^n)\) for \(k=3\). \(O(1,7272^n)\) for \(k=4\).

Recursive formula \(\chi(G)=\min\{\chi(G+uv), \chi(G-uv)\}\) can be used for an
algorithm of order \(O\left({\left(\frac{1+\sqrt{5}}{2}\right)}^{n+m}\right)\).

(proper) edge-coloring: coloring of edges such that incident edges receive
different colors.

edge-chromatic number: minimum number of colors in an edge-coloring. Denoted
\(\chi'(G)\). Observation: \(\chi'(G) \geq \Delta\).

Exercise:
\(\chi'(C_{2k})=2, \chi'(C_{2k+1})=3, \chi'(K_{2n})=2n-1, \chi'(K_{2n+1})=2n+1\).

Theorem (Vizing) \(\chi'(G) \leq \Delta + 1\)

Key idea: consider a partial coloring and let \(P\) be a maximal 2-edge-colored
path (or an even cycle). If we switch the colors on edges of \(P\), the we still
have a partial coloring.

A proof of Vizing theorem (by Ehrenfecht, Fabor, Kierstead).  We aim a proving a
stronger claim by induction.

Assume: \begin{enumerate}
\item \(d(v) \leq k\) (induction on \(k\)),
\item for \(u ~ v, d(u) \leq k\),
\item for at most one \(u ~ v, d(u) = k\),
\item \(G \setminus v\) is \(k\)-edge-colorable,
\end{enumerate} then \(G\) is \(k\)-edge-colorable.

Step 1. By adding pendant vertices, if needed we may assume \(d(u_1) = k, d(u_2)
= d(u_3) =\cdots = k - 1\).

We consider a \(k\)-edge-coloring of \(G \setminus v\) and define
\[\chi_i = {\text{neighbours of } v \text{ which miss color } i}.\]

Among all \(k\)-edge-colorings of \(G\setminus v\) we chose one such that
\(\sum{|x_i|}^2\) is minimized.

\[\sum_{i=1}^k|\chi_i| = 2d(v)-1\]

Claim. For some \(i, 1 \leq i \leq k\) we have \(|\chi_i| = 1\). Otherwise for
some \(i\) we have \(|\chi_i| = 0\) and since \(\sum|\chi_i|\) is an odd number,
for some \(j\) we have \(|\chi_j| \geq 3\). Now consider a maximal
\({i, j}\)-colored path starting with a vertex in \(\chi_j\) (and not in
\(\chi_i\)).  Exchanging the colors \(i\) and \(j\) on this path will/may
increase \(|\chi_i|\) by \(1\) and decrease \(|\chi_j|\) by \(1\), but this
decreases \(\sum{|\chi_i|}^2\), contradiction.

That means a color say \(i\) is missed exactly in one vertex, say \(u\). Remove
from \(G\) the edge \(u v\) and all edges of color \(i\). Let \(G'\) be the
resulting graph. Then \(G'\) and \(v\) satisfy the conditions of our main claim
for \(k-1\).

Corollary. If the set of vertices of degree \(\Delta(G)\) induces an acyclic
graph (a forest), then \(\chi'(G) = \Delta(G)\).

Question: Does this proof translate to a polynomial-time algorithm to find a
\((\Delta + 1)\)-edge-coloring of a give simple graph?

A gadget to build cubic \(2\)-connected graphs that are not
\(3\)-edge-colorable.

Lemma. In any \(3\)-edge-coloring of this graphs either left parallel edges or
the right ones receive a same color and the other \(3\) pendant edges receive
\(3\) distinct colors. Conversely, any such coloring of the \(5\) pendant edges
can be extended to the whole graph.

NP (non-deterministic polynomial time), NP-completeness, NP-hard.

A problem is in NP if every solution can be verified in polynomial time.

A problem is NP-hard if every problem in NP can be reduced to it in polynomial
time.

Three different types of reductions are considered.
\begin{itemize}
\item Many-to-one (Karp reduction);
\item Truth-table;
\item Turing reduction (or Cook-reduction). Cook proved in 1971 that \(3\)-SAT
  is NP-hard.
\end{itemize}

NP-complete: an NP-hard problem which is itself in NP.\@

In our study to prove that a problem \(P\) is NP-hard, we show a many-to-one
reduction from a problem already known to be NP-hard to our problem \(P\).

Problem: \(3\)-edge-coloring (sub)-cubic graph (\(3\)-ECCG).
Input: a subcubic graph \(G\).
Output: Yes --- if \(G\) is \(3\)-edge-colorable, No if \(G\) is not
\(3\)-edge-colorable.

Note: It is enough to consider cubic instances.

\begin{theorem}
  The problem \(3\)-ECCG is NP-complete.
\end{theorem}

\begin{proof}
  Since a coloring can be verified in \(|E(G)|\) steps, the problem is in NP.\@

  To complete the proof we show a polynomial-time reduction from \(3\)-SAT to
  \(3\)-ECCG.\@ That is given an instance \(L\) of \(3\)-SAT, we build a graph
  \(G_L\) such that:

  \begin{enumerate}
  \item \(G_L\) is subcubic;
  \item Order of \(G_L\) is polynomial in size of \(L\) (number of variables and
    clauses of \(L\));
  \item \(L\) is a YES instance of \(3\)-SAT iff \(G_L\) is
    \(3\)-edge-colorable.
  \end{enumerate}

  \[L: (x_1 \vee x_2 \vee \bar{x_3}) \wedge (x_1 \vee \bar{x_2} \vee x_5)
    \wedge \cdots\]

  Variable \(x_i\) appears \(k\) times. Suppose variable \(x_i\) appears \(k\)
  times (in a positive or negative form).  Associate to \(x_i\) the following
  \enquote{variable gadgets}.

  Step 1. Form pairs.
  Step 2. Form a cycle using \(k\) pairs.


  Main property of variable gadgets: in every \(3\)-edge-coloring either each
  paris of pendant edges receive the same color (\(x_i \to \top\)) or none of
  them (\(x_j \to \bot\)).  Moreover, any choice of coloring of pendant edge
  using at most \(3\) colors satisfying this condition can extend to \(3\)-edge
  coloring of the whole graph.

  Variable gadget for \(x_i\) (in \(k\) clauses).
  \(x_i \to \top\): when all pairs are monochromatic.
  \(x_j \to \bot\): when all pairs are \(2\)-colored.

  To complete our proof we must build a clause gadget of the form where a
  \(3\)-edge-coloring is possible iff at least one pairs of pendant edges is
  monochromatic. This is the case in:
\end{proof}

Remarks.  Line graph of \(G\): \(L(G)\), vertices: edges of \(G\), edges: \(e_i
~ e_j\) iff they have a common vertex.

Observation.  Except for \(K_3\), \(w(L(G)) = \Delta(G)\), where \(w\) is the
clique number and \(G\) is a simple graph.

Corollary of Vizing's theorem: for every line graph we have \(w(L(G)) \leq
\chi(L(G)) \leq w(L(G)) + 1\).

Question. Given a graph \(H\) is there a graph \(G\) such that \(H \cong L(G)\)?

\end{document}
